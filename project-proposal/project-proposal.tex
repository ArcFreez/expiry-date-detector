\documentclass[12pt]{article}
\title{Project Proposal}
\begin{document} \maketitle
\textbf{Group Members:} Abhaya Shrestha (\texttt{ashrestha@mines.edu}), and
Jason Rosenblum (\texttt{jhrosenblum@mines.edu}) \\ \\
\textbf{Description} \\
Today, there are often cases where people get ill or die
due to products that are unsafe to consume. Our project will 
detect expiry label on products (particulary
food, and medicine), and determine if it is safe to consume which
will prevent people from getting ill or dying. \\ \\
\textbf{Approach} \\
The approach we plan to take is to use a mixture of techniques from class
to achieve the most accuracte result:
\begin{enumerate}
  \item \textbf{Cleanup step:} first cleanup any distortions
  using camera calibration techiniques (example: the label
  could be a rectangle in a cylindrical product, or could be
  on the side of a jar cap; therefore, this step is absolutely
  crucial) to map the label onto a rectangular image.
  \item \textbf{Detection step:} For each unique colors
  of data we collect (see \textbf{Data} section) that labels are written in do the following.
  \begin{itemize}
      \item Use hue, saturation and vue thresholding technique to detect that color,
      and compute connected components.
      \item Convert the connected components of that color to black, with a white
      background using thresholding.
      \item Use the compiled list of keyboard letter templates written
      in black to match the letters in the processed image (see \textbf{Data} section).
      \item See if the template matching spells any of the starting substring
      collected in data (see \textbf{Data} section). If it does, then that means we have found the
      substring, and we can use it's geometry to find matched templates to the
      right of that substring which indicates date.
  \end{itemize}
  \item \textbf{Evaulation of step:} If the date label could
  be detected and date format be understood, output
  whether or not it is safe to consume. Otherwise, if the steps above for detection could not
  detect it or if date format could not be understood, output not detected.
\end{enumerate}
\textbf{Data} \\
For our algorithm to work properly, we need to collect data on
\begin{enumerate}
    \item The unique color labels for food and medicine. This will be used
    in our detection step for HSV thresholding technique.
    \item Limiting the templates to the English keyboard letters, we need to
    gather the letters written in black for the most common
    fonts that most food/medicine labels are written in.
    \item The unique labels substring for expirey labels (example: ``used by'', ``expires by'' etc.)
    \item A dataset of images of food and medicine labels which contain
    the label or which do not contain the label (we also need to test true positive,
    and true negatives). We can gather these from retail stores, or from
    online pictures that are free to use.
\end{enumerate}
\textbf{Timeline}
\begin{itemize}
    \item \textbf{November 21st:} finish collecting data, finish research on
    possible inputs our algorithm (mostly going to be
    the variety of labels we compile and the unique colors of
    labels our algorithm needs to detect) needs to process,
    and start working on implementation.
    \item \textbf{November 28th:} have presentation finished, and
    continue working on implementation.
    \item \textbf{December 9th:} work
    on optimizing algorithm accuracy.
    \item \textbf{December 12th:} have code finished, and have paper finished.
\end{itemize}

\end{document}
